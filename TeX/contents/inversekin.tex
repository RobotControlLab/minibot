\section{Inverse Kinematics (IK)}
\label{sec:inversekin}

Suppose that we are given a point $H \in \SEthree$: 
\[ \bm{H} = \bmat{\bm{R} & \bm{o} \\ 0 & 1}. \]
We want to find a $\bm{q} \in \mc{Q}$ such that $f(\bm{q}) = \bm{H}$. We
describe two methods to achieve this, which would yield equivalent solutions for
a fully-actuated robot, but the differences get amplified when the robot is
redundant such as miniBot-7R.

\subsection{IK by Decoupling Position and Orientation}
\label{ssec:ik_decoupling}

We first consider the inverse position kinematics problem, 
which assumes that the inverse orientation kinematics may be 
solved by using the final three joints, i.e., whatever the 
rotation matrix $^0\bm{R}_4$ is, there exist $q_5$, $q_6$ and $q_7$, 
such that ${}^0\bm{R}_4 {}^4\bm{R}_7 = R$. The inverse position kinematics 
problem is then to find $q_1$ through $q_4$ such that the product $\prod_1^4
\bm{A}_i$ has as its translation vector the wrist center location, $\bm{o}_c \in
\mathbb{R}^3$.

In order to solve the inverse position problem, we first find the location of
the wrist center $\bm{o}_c$ in the base coordinate system,
$\Sigma_0$: \[ \bm{o}_c = \bm{o} - d_7 \bm{R} \bmat{0 \\ 0 \\ 1}. \]
%
We denote the current guess to the solution $\bm{q}^d$ to the inverse position
kinematics problem by $\bm{q}^{(k)}$ and perform the following iteration.

\begin{align}
  \begin{split}
  \bar{\bm{q}}^{(k+1)} &= \bm{q}^{(k)} - \alpha_k \bm{M}\left(\bm{q}^{(k)}
  \right)\left(f\left(\bm{q}^{(k)}\right) - \bm{o}_c \right), \\
  \bm{q}^{(k+1)} &= \operatorname{clamp}\left(\bar{\bm{q}}^{(k+1)}, \bm{q}_{\text{lb}}, \bm{q}_{\text{ub}} \right).
  \end{split}
  \label{eq:ipk_iter}
\end{align}
%
where the matrix $\bm{M}$ could be taken as the pseudo-inverse or the transpose
of the Jacobian, $\bm{J}_w$, that maps the rates of changes $\dot{q}_1$ through
$\dot{q}_4$ to the rate of change of the wrist center, i.e., $\bm{J}_w
\dot{\bm{q}}_{1:4} = \dot{\bm{o}}_c$. This matrix $\bm{J}_w$ corresponds to the 
first four columns of the matrix $\bm{J}_v$ from equation~\eqref{eq:jac_v}.
%
Notice that the step size $\alpha_k$ may be orders of magnitude different
depending on whether $\bm{M} = \bm{J}_w^\top$ or $\bm{M} = \bm{J}_w^\dagger$. 


\subsection{Fully Coupled IK}
\label{ssec:ik_full}
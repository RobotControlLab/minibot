\section{Inverse Kinematics}
\label{sec:inversekin}

Suppose that we are given a point $H \in \SEthree$: 
\[ H = \bmat{R & o \\ 0 & 1}. \]
We want to find a $q \in \mc{Q}$ such that $f(q) = H$.

We first consider the inverse position kinematics problem, 
which assumes that the inverse orientation kinematics may be 
solved by using the final three joints, i.e., whatever the 
rotation matrix $^0R_4$ is, there exist $q_5$, $q_6$ and $q_7$, 
such that ${}^0R_4 {}^4R_7 = R$. The inverse position kinematics 
problem is then to find $q_1$ through $q_4$ such that the product $\prod_1^4
A_i$ has as its translation vector the wrist center location, $o_c \in
\mathbb{R}^3$.

In order to solve the inverse position problem, we first find the location of
the wrist center $o_c$ in the base coordinate system,
$\Sigma_0$: \[ o_c = o - d_7 R \bmat{0 \\ 0 \\ 1}. \]
%
We denote the current guess to the solution $q^d$ to the inverse position
kinematics problem by $q^{(k)}$ and perform the following iteration.

\begin{equation}
  q^{(k+1)} = q^{(k)} - \alpha_k M\left(q^{(k)}\right)\left(f\left(q^{(k)}\right) - o_c \right),
  \label{eq:ipk_iter}
\end{equation}
%
where the matrix $M$ could be taken as $\bar{J}_v^\dagger$ or $\bar{J}_v^\top$. 
$\bar{J}_v$ is the translation part of the Jacobian matrix that relates
$\dot{q}_1$ through $\dot{q}_4$ to the rate of change of the wrist center, i.e., 
$\bar{J}_v \dot{q}_{1:4} = \dot{o}_c$, where

\[ \bar{J}_v = \bmat{\bar{J}_{v_1} & \cdots & \bar{J}_{v_4}}, \]
where $\bar{J}_{v_{i+1}} = z_i \times (o_c - o_i)$, for $i = 0, \ldots 3$.
